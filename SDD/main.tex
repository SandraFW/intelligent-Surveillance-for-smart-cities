\documentclass[12pt]{article}
\usepackage{graphicx}
\usepackage[utf8]{inputenc}
\usepackage{makecell}
\usepackage[margin=1in]{geometry}
\usepackage[T1]{fontenc}
\usepackage{mathptmx}
\usepackage[normalem]{ulem}
\usepackage{placeins}

\renewcommand\theadalign{bc}
\renewcommand\theadfont{\bfseries}
\renewcommand\theadgape{\Gape[4pt]}
\renewcommand\cellgape{\Gape[4pt]}
\graphicspath{ {./images/} }
\usepackage[
backend=biber,
style=numeric,
sorting=none,
citestyle=ieee
]{biblatex}

\addbibresource{bibliography.bib}

%opening
\title{Software Design Description Document for Intelligent Surveillance system for smart cities}
\author{
Nour Ahmed, Nour El Hoda Hisham, Mariam Hesham, Samiha Hesham ,Sandra Fares\\
Supervised by: Eng. Lobna Mostafa,  Dr. Islam Tharwat
}
\begin{document}
\maketitle

begin{table}[htbp]
\caption{Document version history}
\begin{center}
\begin{tabular}{|c|c|l|}
\hline
\thead{Version}    & \thead{Date} & \thead{Reason for Change}  \\ \hline
1.0 & 25-Mar-2021   & \makecell[l]{SDD first version’s description are defined.}   \\ \hline
1.1 & 20-Mar-2021   & \makecell[l]{Added Sequence Diagram.} \\ \hline
1.2 & 25-Mar-2021   & \makecell[l]{Requirement Matrix updated.} \\
\hline
2.0 & 29-Mar-2021   & \makecell[l]{Diagrams updated.} \\
\hline
\end{tabular}
\end{center}
\end{table}
\begin{table}[htbp]
\begin{tabular}{cc}
\thead{GitHub:}    & {https://github.com/SandraFW/intelligent-Surveillance-for-smart-cities}   
\end{tabular}
\end{table}

\pagebreak
\tableofcontents
\pagebreak

\begin{abstract}
Add your project abstract here.
(Word Limit 150)
\end{abstract}

\section{Introduction}

\subsection{Purpose}
The purpose of the software design document (SDD) is to present the project ” Intelligent Surveillance for smart cities” thoroughly. SSD illustrates the framework and system design architecture of the proposed system. Moreover, the document provides a fully detailed system overview and scope besides stating the main objective of the system. The SDD document is considered a significant guide for the project team members and for the customer to be fully aware of the project. Also, for the developers to improve and develop the system in the future.

\subsection{Scope}

The scope of the software design document (SDD) is to describe the project ” Intelligent Surveillance for smart cities” in detail. The SDD provides the main design of the data and viewpoints to communicate to key design stakeholders. Also, it states the main objective of the system with a well-studied timeline. Besides, the algorithms and human Interface design that utilized in the project. Moreover, the states of the functional requirements of the system.

\subsection{Overview}
The software design document (SDD) presents the overall system architecture in addition to design patterns, data models, data structures, algorithms, and design viewpoints. Moreover to the human interface design to know how to utilize the provided services correctly. The software design document is structured as follows.

 Section I introduces the document by defining its purpose, scope, and overview. In addition to the audience that the document is intended for. Section II presents the system thoroughly by illustrating the system overview and describing the functionality of the whole system. In addition to the system scope and brief description of system main features and boundaries. Moreover, it states the main objective of the system with the project plan. Section III illustrates the design viewpoints of the system. Section Iv presents the data design for the system. It explains the information related to the data in the system and how it is stored. In addition to the thorough description of the dataset and database utilized in the system. Section V presenting the human interface design for the system. Section VI describes the requirements matrix that clarifies the status of the functional requirements mentioned in the SRS. Section VII stating the appendices that could help in well understanding the document.

\subsection{Intended audience}

\begin{itemize}
    \item Stakeholders: 
    Dr. Islam Tharwat,  
    Eng. Lobna Shaheen,   
    Nour Ahmed Ghoneim,  
    Samiha Hesham,
    Nour El-Hoda Hesham,  
    Mariam Hesham, 
    Sandra Fares.
    
\item User: 
    Police Officers,  
    Admin  
    
\item Clients:  
    Police,
    Security Government Companies, 
    El SEWEDY technology company.  

\end{itemize}



\subsection{Reference Material}
We have done a  SRS which is mainly talking about how to use the edge-fog model and  Proposal to handle abnormal movements in streets like kidnapping, fighting, and robbery.  Moreover, we have done a proposal which is talking about big data and how to handle it by saving important data only and increase the security of this data from any massive or malicious action.


\begin{figure}[htbp]
\centering
\includegraphics[width=0.9\linewidth]{proposal.png}
\caption{Proposal}
\label{archdsgn}
\end{figure}
\FloatBarrier

\begin{figure}[htbp]
\centering
\includegraphics[width=0.9\linewidth]{SRS.png}
\caption{SRS}
\label{archdsgn}
\end{figure}
\FloatBarrier


\subsection{Definitions and Acronyms}

\begin{table}[htbp]
\begin{tabular}{|p{0.35\textwidth}|p{0.85\textwidth}|}
\hline
\multicolumn{1}{|c|}{\textbf{Term}} & \multicolumn{1}{c|}{\textbf{Definition}}                                                                                              \\ \hline
EFISS                      & \begin{tabular}[c]{@{}l@{}}Edge Fog intelligent surveillance system.\end{tabular} \\ \hline
Graphical User Interface (GUI)      & \begin{tabular}[c]{@{}l@{}}is a form of user interface that allows user to interact with electronic devices\\ through graphical icons and audio indicator .\end{tabular}                  \\ \hline

Design Rationale             & \begin{tabular}[c]{@{}l@{}}Information capturing the reasoning of the designer that \\led to the system as designed, including design options, \\trade-offs considered, decisions made, \\and the justifications of those
decisions. .\end{tabular} \\ \hline

\end{tabular}
\end{table}
 
\section{System Overview}


\subsection{System Scope}

Our main objective is to build a fog computing based framework for surveillance data in order to detect abnormal human behaviors in the streets(fighting,kidnapping,robbery). The system is also developed to work on enhancing the latency by reducing the amount of data. Moreover, we will work on developing the privacy protection scheme for data security.

\begin{enumerate}
\item Fog/Edge computing-based surveillance system to monitor the whole city 24/7.

\item Privacy assurance by blurring people’s faces to secure their exposed identity in the video.
\item Scalability system to handle such massive data generated from surveillance cameras and sensors.
\item Efficient image processing techniques for certain failures in the edge cameras that impacts the usefulness of the video streams.
\item Dispatch crime details (i.e., video and location of the event, criminal’s face) and giving alert to the nearest competent authority.



\end{enumerate} 

\subsection{System objectives}
\begin{enumerate}


\item The system will be able to detect the action occurred accurately.

\item The system should apply blurring method for normal citizen’s bio-metrics in order to enhance privacy in surveillance system.
\item The system should send alert if any failure is detected in the cameras in order to help the maintenance team to solve the problem.
\item The system will send alert message when anomaly behaviour is detected using the fog nodes in the architecture system.

\item The system will filter the incident using video splitting technique in order to be saved and sent to the cloud to save storage.

\item The system should provide face recognition feature for the kidnapper and the victim to get their information.
\item The system ensures usability to achieve the demand efficiently with low latency and high bandwidth.
\item The Fog and edge models will reduce transmitted data which makes it easier for the system to reprocessing it, increase usability, and increase the potential of the cloud platform.

 \end{enumerate}
 \newpage
\subsection {System Timeline}

\begin{table}[htbp]
\centering
\caption{ISS time plan}
\label{tab:Time Plan}


\begin{tabular}{|l|l|l|l|l|}
\hline
Id & Task               & Start Date & Number of Days & Team Member \\ \hline
1  & Collecting Dataset    & 2/1/2021 & 20           & nour ahmed, sandra        \\ \hline
2  & Work on GUI        & 3/1/2021 & 15             & samiha, mariam        \\ \hline
3  & camera failure    & 3/5/2021 & 30              & nour el hoda          \\ \hline
4  &  anomaly detection & 3/1/2021 &20             & sandra    \\ \hline
5  & video summarization     & 3/1/2021 & 20             & nour ahmed           \\ \hline

6 & crime classification     & 2/4/2021 & 10    & nour ahmed, sandra          \\ \hline
7 &   face blurring technique  & 3/20/2021 & 20             & mariam     \\ \hline
8 &   face detection    & 3/20/2021 & 20             & samiha    \\ \hline
9 &  deep learning detection classifier    & 4/5/2021 & 20             & nour ahmed,sandra    \\ \hline
10 & mobile application    & 4/5/2021 & 20             & samiha,mariam    \\ \hline
11 & enhancing video quality   & 4/5/2021 & 15             & nour elhoda          \\ \hline

\hline
\end{tabular}
\end{table}

\begin{figure}[htbp]
\centering
\includegraphics[width=\linewidth]{Taskplan.PNG}
\caption{ISS GANTT Chart}
\label{fig:GANTT Chart}
\end{figure}

\clearpage
\section{Design viewpoints} 
\subsection{ Context viewpoint}
The intelligent surveillance system monitors smart city streets to assists the police units to get notified of any incident that occurs to take quick action against it using the three computing layers Edge, Fog and cloud. Our proposed system aims to provide a high level of safety to the citizens. Therefore the system captures the pedestrians to collect data throughout the day. The collected data are processed to detect any failure in the surveillance cameras or whenever an abnormal activity occurs, it captures the incident, then identifies whether it is a crime or not; if it is, the system identifies the criminal and forwards the collected information to the police unit to take the proper action and helps them track the criminal.

\begin{figure}[htbp]
\centering
\includegraphics[width=0.8\linewidth]{contextdiagram.jpeg}
\caption{Context Diagram for the Edge Fog intelligent surveillance system}
\label{fig:ctx}
\end{figure}
\FloatBarrier

\begin{figure}[htbp]
\centering
\includegraphics[width=0.9\linewidth]{survillienceusecase.png}
\caption{Use Case Diagram Example}
\label{fig:ucase}
\end{figure}
\FloatBarrier
\subsection{Composition viewpoint}\label{ss-comp}

Our system is composed of three layers Edge, Fog, and cloud layers; Each of them has responsibilities, and they communicate together through a stable network to provide the user at the end the needed result.
\begin{itemize}
\item Edge Layer: The first layer in our system is the surveillance cameras fixed within the smart street lights. Those cameras are responsible for capturing the pedestrians throughout the day and using a crime detection algorithm whenever any abnormal actions are recognized the video stream is transferred to our second layer; and if everything is normal, the videos are being stored on the camera's local storage and then sent to the cloud when the network is free. Also, we consider any failure in cameras using a failure detection algorithm to avoid any blurry recording, green or black videos being recorded. Finally, there is a comparing algorithm that helps tracking wanted criminals whenever detected in any street the authorities are informed.
\item Fog Layer: The second layer in our system, This layer is divided into two sub-layers; the first depends on connected vehicles near the incident and the second is smart traffic light which works as gateways.
\subitem The vehicles layer: It is responsible for extracting more features from the video stream to be more sure about the incident and summarize it into an image using a special algorithm for further processing in the second fog sub-layer.
\subitem The smart traffic light: It is responsible for the final processing of the incident using a deep learning algorithm; to ensure the incident, classify it and send it to the cloud layer the final layer. Also, an image processing deblurring algorithm is applied for face detection. then the criminal face image is sent back to the edge nodes.
\item Cloud Layer: The final layer; is responsible for triggering the alarms and send notifications for our end users, and backup all the incident data sent. Also, it is used for archiving all the videos captured throughout the day.
\item All the layers should communicate together using a stable wireless network provided in the smart city.
\end{itemize} 

\begin{figure}[htbp]
\centering
\includegraphics[width=0.9\linewidth]{arch.jpeg}
\caption{Architectural Design Diagram }
\label{archdsgn}
\end{figure}
\FloatBarrier
\textbf{Design concerns:}
Main design concerns are focused on robust and efficient communication between each computing layer. We put into consideration the computational power of devices used in each layer to be compatible with the functions deployed to them. performance of the functions, liable deep learning operation results, and avoiding false-negative detection also was of the main concerns to avoid false alarms.Finally, the system should follow ethics and protects the privacy of the citizens.

\subsubsection{Design Rationale}
Various architectures have been introduced in the field of real-time video surveillance.
\begin{itemize}
\item Cloud Computing: At first, systems were implemented using only a cloud computing layer; But they were expensive, relatively slow to perform real-time video surveillance, and there was consistently heavy traffic on the network. 
\item Edge or Fog layers and cloud Computing : Therefore, they started adding fog layers or edge layers individually, where the performance was enhanced; but it was introduced for smaller areas as smart homes, also the devices used needed high computational power or less powerful functions were used.
\end{itemize}
\newline  So, We built our architecture that comprises all three layers to get maximum advantages from each layer; to easily monitor the whole smart city. Each layer performs specific functions either simple image processing or deep learning algorithms, therefore most of the false alarms were reduced; those functions were chosen to be deployed according to each layer's components' computational power and to efficiently serve the proper flow of the system. Then each layer communicates with others through a stable network to forward and process the results in case of detecting an incident only; thus the network traffic was reduced. And finally notifying our end users of those incidents efficiently.
\FloatBarrier
\subsection{Logical viewpoint}

\subsubsection{Class Diagram}
 \FloatBarrier
\begin{figure}[htbp]
\centering
\includegraphics[width=0.85\linewidth]{classdiagram.png}
\caption{ClassDiagram }
\label{archdsgn}
\end{figure}
\subsubsection{Detailed classes}
\begin{table}[h!]

\caption{Class Name - officers}
\label{tab:my-table}
\begin{tabular}{|p{0.15\textwidth}|p{0.85\textwidth}|}

 \hline
\textbf{List of Superclasses}  & none                                                      
\\ \hline
\textbf{List of Subclasses}    & none                                                                    
\\ \hline
\textbf{Purpose}               & The class represents police officer'data                                                                             
\\ \hline
\textbf{Collaborations}        & associated with shift type,stations,department and aggregated with officers type and notifications.
\\ \hline
\textbf{Attributes}  & officerid:int,officerTypeID:int,firstname:string,last name:string,email:string,
password:string,stationID:int,deployment date:date,shiftID:int,isdeleted:boolean,
notifications[]:notification.
\\ \hline
\textbf{Operations} & 
login(email,password),logout(),signUP(firstname,lastname,email,password,
officersTypeID),updateINFO(officersID,firstname,lastname,email,password),
viewNotifications(notificationID),searchNotifications().

\\ \hline
\textbf{Constraints} & officer is one of the core classes of the application  
\\ \hline
\end{tabular}
\end{table}

\begin{table}[h!]
\caption{Class Name - Officers type}
\label{tab:my-table}
\begin{tabular}{|p{0.15\textwidth}|p{0.85\textwidth}|}


 \hline
\textbf{List of Superclasses}  & none                                                    
\\ \hline
\textbf{List of Subclasses}    & none                                                                    
\\ \hline
\textbf{Purpose}               & the class represents officer's types.                                                                           
\\ \hline
\textbf{Collaborations}        & aggregated with officers.
\\ \hline
\textbf{Attributes}  & ID:int,officerTypeID:int,isdeleted:boolean
\\ \hline
\textbf{Operations} & 
getusertypeID(),insert(),edit(),delete()

\\ \hline
\textbf{Constraints} & cannot work without officers class
\\ \hline
\end{tabular}
\end{table}

\begin{table}[h!]
\caption{Class Name - Admin}
\label{tab:my-table}
\begin{tabular}{|p{0.15\textwidth}|p{0.85\textwidth}|}


 \hline
\textbf{List of Superclasses}  & officers                                                     
\\ \hline
\textbf{List of Subclasses}    & none                                                                    
\\ \hline
\textbf{Purpose}               & a class represent admin in the system                                                                           
\\ \hline
\textbf{Collaborations}        & extends with officers.
\\ \hline
\textbf{Attributes}  & ID:int,email:string,password:string
\\ \hline
\textbf{Operations} & 
createpoliceofficer(),edit(),delete(),checkFailure(),AddFognodes()

\\ \hline
\textbf{Constraints} & Admin is the main controller in the application.
\\ \hline
\end{tabular}
\end{table}


\begin{table}[h!]
\caption{Class Name - shift Type}
\label{tab:my-table}
\begin{tabular}{|p{0.15\textwidth}|p{0.85\textwidth}|}


 \hline
\textbf{List of Superclasses}  & none                                                     
\\ \hline
\textbf{List of Subclasses}    & none                                                                    
\\ \hline
\textbf{Purpose}               & the class represents shift types for the police officers.                                                                        
\\ \hline
\textbf{Collaborations}        & associated with officers.
\\ \hline
\textbf{Attributes}  & ID:int,shiftID:int,officerID:int,shiftTimeStarted:datetime,shiftTimeEnded
:datetime,isdeleted:boolean
\\ \hline
\textbf{Operations} & 
getShiftTypeID(),insert(),edit(),delete()
\\ \hline
\textbf{Constraints} & can't work without officers.
\\ \hline
\end{tabular}
\end{table}


\begin{table}[h!]
\caption{Class Name - department}
\label{tab:my-table}
\begin{tabular}{|p{0.15\textwidth}|p{0.85\textwidth}|}


 \hline
\textbf{List of Superclasses}  & none                                                     
\\ \hline
\textbf{List of Subclasses}    & none                                                                    
\\ \hline
\textbf{Purpose}               & the class represents departments of the police officers.                                                                        
\\ \hline
\textbf{Collaborations}        & associated with officers.
\\ \hline
\textbf{Attributes}  & depCode:int,depDescription:string
\\ \hline
\textbf{Operations} & 
getDepCode(),insert(),edit(),delete()
\\ \hline
\textbf{Constraints} & cannot work without officers.
\\ \hline
\end{tabular}
\end{table}

\begin{table}[h!]
\caption{Class Name - stations}
\label{tab:my-table}
\begin{tabular}{|p{0.15\textwidth}|p{0.85\textwidth}|}


 \hline
\textbf{List of Superclasses}  & none                                                     
\\ \hline
\textbf{List of Subclasses}    & none                                                                    
\\ \hline
\textbf{Purpose}               & the class represents stations locations of each police officer.                                                                        
\\ \hline
\textbf{Collaborations}        & associated with officers and locations.
\\ \hline
\textbf{Attributes}  & stationID:int,locationID:int
\\ \hline
\textbf{Operations} & 
getLocation()
\\ \hline
\textbf{Constraints} & cannot work without officers and locations.
\\ \hline
\end{tabular}
\end{table}


\begin{table}[h!]
\caption{Class Name - notifications}
\label{tab:my-table}
\begin{tabular}{|p{0.15\textwidth}|p{0.85\textwidth}|}


 \hline
\textbf{List of Superclasses}  & none                                                     
\\ \hline
\textbf{List of Subclasses}    & none                                                                    
\\ \hline
\textbf{Purpose}               & the class represents the notifications that views to the police officer.                                                                        
\\ \hline
\textbf{Collaborations}        & associated with fog nodes and aggregated with officers.
\\ \hline
\textbf{Attributes}  & ID:int,message:string,FogNodeID:int
\\ \hline
\textbf{Operations} & saveLocation(),sendNotification(),getNearestStation()
getLocation()
\\ \hline
\textbf{Constraints} & cannot work without officers and fognodes.
\\ \hline
\end{tabular}
\end{table}


\begin{table}[h!]
\caption{Class Name - fog nodes}
\label{tab:my-table}
\begin{tabular}{|p{0.15\textwidth}|p{0.85\textwidth}|}


 \hline
\textbf{List of Superclasses}  & none                                                     
\\ \hline
\textbf{List of Subclasses}    & none                                                                    
\\ \hline
\textbf{Purpose}               & the class represents the fognodes that sends the alerts that came from the edge node to the officers through class notifications                                                                        
\\ \hline
\textbf{Collaborations}        & associated with notifications,locations and cameras.
\\ \hline
\textbf{Attributes}  & FogNodeID:int,data:string
\\ \hline
\textbf{Operations} & prediction(),detection(),trackingLocation()
\\ \hline
\textbf{Constraints} & cannot work without locations and cameras.
\\ \hline
\end{tabular}
\end{table}


\begin{table}[h!]
\caption{Class Name - cameras}
\label{tab:my-table}
\begin{tabular}{|p{0.15\textwidth}|p{0.85\textwidth}|}


 \hline
\textbf{List of Superclasses}  & none                                                     
\\ \hline
\textbf{List of Subclasses}    & none                                                                    
\\ \hline
\textbf{Purpose}               & the class represents the edge nodes that carries the video data and ready to be sent to fog nodes.                                                                       
\\ \hline
\textbf{Collaborations}        & associated with locations,fognodes and videos.
\\ \hline
\textbf{Attributes}  & cameraID:int,noOfcameras:int,locationID:int
,serialNumber:int
\\ \hline
\textbf{Operations} & getVideos(),sendToFognodes(),caameraFailure(),CrimeDetection()
\\ \hline
\textbf{Constraints} & cannot work without videos and locations.
\\ \hline
\end{tabular}
\end{table}

\begin{table}[h!]
\caption{Class Name - locations}
\label{tab:my-table}
\begin{tabular}{|p{0.15\textwidth}|p{0.85\textwidth}|}


 \hline
\textbf{List of Superclasses}  & none                                                     
\\ \hline
\textbf{List of Subclasses}    & none                                                                    
\\ \hline
\textbf{Purpose}               & the class represents the locations of the surveillance camera that the kidnapping event will occur.                                                                
\\ \hline
\textbf{Collaborations}        & associated with cameras,stations,fogNodes
and event.
\\ \hline
\textbf{Attributes}  & LocationID:int,noOfcameras:int,street:string,district:string
\\ \hline
\textbf{Operations} & getLocation(),sendLocation()
\\ \hline
\textbf{Constraints} & cannot work without event class
\\ \hline
\end{tabular}
\end{table}


\begin{table}[h!]
\caption{Class Name - video}
\label{tab:my-table}
\begin{tabular}{|p{0.15\textwidth}|p{0.85\textwidth}|}


 \hline
\textbf{List of Superclasses}  & none                                                     
\\ \hline
\textbf{List of Subclasses}    & none                                                                    
\\ \hline
\textbf{Purpose}               & the class represents video data.                                                               
\\ \hline
\textbf{Collaborations}        & associated with cameras,event and aggregated with object.
and event.
\\ \hline
\textbf{Attributes}  & id:int,title:string,duration:double,dateCreated:date,sizeInBytes:int,path:string
\\ \hline
\textbf{Operations} & getSequences(),sendObjects()
\\ \hline
\textbf{Constraints} & cannot work without object class.
\\ \hline
\end{tabular}
\end{table}

\begin{table}[h!]
\caption{Class Name - object}
\label{tab:my-table}
\begin{tabular}{|p{0.15\textwidth}|p{0.85\textwidth}|}


 \hline
\textbf{List of Superclasses}  & none                                                     
\\ \hline
\textbf{List of Subclasses}    & actors                                                                   
\\ \hline
\textbf{Purpose}               & the class represents objects in the video frame.                                                               
\\ \hline
\textbf{Collaborations}        & associated with region and aggregated with video class. 
\\ \hline
\textbf{Attributes}  & id:int,description:string
\\ \hline
\textbf{Operations} & getVideo(),getFrame(),getRegion(),faceDetection(),faceBlurring
\\ \hline
\textbf{Constraints} & cannot work without video class.
\\ \hline
\end{tabular}
\end{table}


\begin{table}[h!]
\caption{Class Name - actors}
\label{tab:my-table}
\begin{tabular}{|p{0.15\textwidth}|p{0.85\textwidth}|}


 \hline
\textbf{List of Superclasses}  & object                                                    
\\ \hline
\textbf{List of Subclasses}    & none                                                                
\\ \hline
\textbf{Purpose}               & the class represents actors which is mainly the kidnapper or the victim in the video frame.                                                               
\\ \hline
\textbf{Collaborations}        & associated with position and event. 
\\ \hline
\textbf{Attributes}  & id:int,location:string
\\ \hline
\textbf{Operations} & getEvent(),getPosition()
\\ \hline
\textbf{Constraints} & cannot work without object class.
\\ \hline
\end{tabular}
\end{table}

\begin{table}[h!]
\caption{Class Name - Region}
\label{tab:my-table}
\begin{tabular}{|p{0.15\textwidth}|p{0.85\textwidth}|}


 \hline
\textbf{List of Superclasses}  &    none                                               
\\ \hline
\textbf{List of Subclasses}    & none                                                                  
\\ \hline
\textbf{Purpose}               & the class represents the region of the objects that are in the frame video.                                                             
\\ \hline
\textbf{Collaborations}        & associated with position and object. 
\\ \hline
\textbf{Attributes}  & topLeftX:int,topLeftY:int,width:int,height:int
\\ \hline
\textbf{Operations} & getter(),setter()
\\ \hline
\textbf{Constraints} & cannot work without object and position classes.
\\ \hline
\end{tabular}
\end{table}

\begin{table}[h!]
\caption{Class Name - position}
\label{tab:my-table}
\begin{tabular}{|p{0.15\textwidth}|p{0.85\textwidth}|}


 \hline
\textbf{List of Superclasses}  &    none                                               
\\ \hline
\textbf{List of Subclasses}    & none                                                                  
\\ \hline
\textbf{Purpose}               & the class represents the exact video frame that the event occur at  be saved and detection will be processed on it.                                      
\\ \hline
\textbf{Collaborations}        & associated with region and actors. 
\\ \hline
\textbf{Attributes}  & startingFrame:double,endingFrame:double
\\ \hline
\textbf{Operations} & getRegion(),getActors()
\\ \hline
\textbf{Constraints} & cannot work without actors class.
\\ \hline
\end{tabular}
\end{table}

\begin{table}[h!]
\caption{Class Name - event}
\label{tab:my-table}
\begin{tabular}{|p{0.15\textwidth}|p{0.85\textwidth}|}


 \hline
\textbf{List of Superclasses}  &    none                                               
\\ \hline
\textbf{List of Subclasses}    & none                                                                  
\\ \hline
\textbf{Purpose}               & the class represents the event of kidnapping.
\\ \hline
\textbf{Collaborations}        & associated with video,actors,locations and aggregated with event status and structure 
\\ \hline
\textbf{Attributes}  & statusCode:id,where:string,when:date,startingFrame:double,endingFrame:double.
\\ \hline
\textbf{Operations} & getshots(),getActors(),getTimeInterval()
\\ \hline
\textbf{Constraints} & cannot work without actors and locations classes.
\\ \hline
\end{tabular}
\end{table}


\begin{table}[h!]
\caption{Class Name - event status}
\label{tab:my-table}
\begin{tabular}{|p{0.15\textwidth}|p{0.85\textwidth}|}


 \hline
\textbf{List of Superclasses}  &    none                                               
\\ \hline
\textbf{List of Subclasses}    & none                                                                  
\\ \hline
\textbf{Purpose}               & the class represents the status of the event occurred.
\\ \hline
\textbf{Collaborations}        & aggregated with event class. 
\\ \hline
\textbf{Attributes}  & statusCode:id,statusDescription:string
\\ \hline
\textbf{Operations} & getter(),setter()
\\ \hline
\textbf{Constraints} & cannot work without event class.
\\ \hline
\end{tabular}
\end{table}


\begin{table}[h!]
\caption{Class Name - structure}
\label{tab:my-table}
\begin{tabular}{|p{0.15\textwidth}|p{0.85\textwidth}|}


 \hline
\textbf{List of Superclasses}  &    none                                               
\\ \hline
\textbf{List of Subclasses}    & scene,sequence,shot                                                                  
\\ \hline
\textbf{Purpose}               & the class represents the structure of the event occurred that is divided into shots,scenes and sequence.
\\ \hline
\textbf{Collaborations}        & aggregated with event class. 
\\ \hline
\textbf{Attributes}  & startingFrame:double,endingFrame:double
\\ \hline
\textbf{Operations} & getTimeInterval(Event)
\\ \hline
\textbf{Constraints} & cannot work without event class.
\\ \hline
\end{tabular}
\end{table}

\begin{table}[h!]
\caption{Class Name - sequence}
\label{tab:my-table}
\begin{tabular}{|p{0.15\textwidth}|p{0.85\textwidth}|}


 \hline
\textbf{List of Superclasses}  &    structure                                              
\\ \hline
\textbf{List of Subclasses}    & none                                                                 
\\ \hline
\textbf{Purpose}               & the class represents the sequence of the structure the video surveillance captured when an event occured.
\\ \hline
\textbf{Collaborations}        & aggregated with scene class. 
\\ \hline
\textbf{Attributes}  & id:int,description:string
\\ \hline
\textbf{Operations} & getScenes()
\\ \hline
\textbf{Constraints} & cannot work without structure class.
\\ \hline
\end{tabular}
\end{table}

\begin{table}[h!]
\caption{Class Name - scene}
\label{tab:my-table}
\begin{tabular}{|p{0.15\textwidth}|p{0.85\textwidth}|}


 \hline
\textbf{List of Superclasses}  &    structure                                              
\\ \hline
\textbf{List of Subclasses}    & none                                                                 
\\ \hline
\textbf{Purpose}               & the class represents the scenes of the kidnapping event that occurred.
\\ \hline
\textbf{Collaborations}        & aggregated with shots class. 
\\ \hline
\textbf{Attributes}  & id:int,description:string
\\ \hline
\textbf{Operations} & getSequence(),getShots()
\\ \hline
\textbf{Constraints} & cannot work without structure class.
\\ \hline
\end{tabular}
\end{table}

\begin{table}[h!]
\caption{Class Name - shots}
\label{tab:my-table}
\begin{tabular}{|p{0.15\textwidth}|p{0.85\textwidth}|}


 \hline
\textbf{List of Superclasses}  &    structure                                              
\\ \hline
\textbf{List of Subclasses}    & none                                                                 
\\ \hline
\textbf{Purpose}               & the class represents the shots of the actors in the event that occurred.
\\ \hline
\textbf{Collaborations}        & associated with frame class. 
\\ \hline
\textbf{Attributes}  & id:int,shotTime:time
\\ \hline
\textbf{Operations} & getFrames(),getActorShots()
\\ \hline
\textbf{Constraints} & cannot work without structure class.
\\ \hline
\end{tabular}
\end{table}

\begin{table}[h!]
\caption{Class Name - Frame}
\label{tab:my-table}
\begin{tabular}{|p{0.15\textwidth}|p{0.85\textwidth}|}


 \hline
\textbf{List of Superclasses}  &    none                                             
\\ \hline
\textbf{List of Subclasses}    & none                                                                 
\\ \hline
\textbf{Purpose}               & the class represents the frame of the actors that occurred in an event. 
\\ \hline
\textbf{Collaborations}        & associated with shot and image classes. 
\\ \hline
\textbf{Attributes}  & FrameNumber:int
\\ \hline
\textbf{Operations} & getImage(),getObject()
\\ \hline
\textbf{Constraints} & cannot work without shot class.
\\ \hline
\end{tabular}
\end{table}


\begin{table}[h!]
\caption{Class Name - Image}
\label{tab:my-table}
\begin{tabular}{|p{0.15\textwidth}|p{0.85\textwidth}|}


 \hline
\textbf{List of Superclasses}  &    none                                             
\\ \hline
\textbf{List of Subclasses}    & none                                                                 
\\ \hline
\textbf{Purpose}               & the class represents the image data of the frame captured.
\\ \hline
\textbf{Collaborations}        & associated with frame class. 
\\ \hline
\textbf{Attributes}  & title:string,format:string,width:int,height:int,texture:string,
histogram:string,rawData:string
\\ \hline
\textbf{Operations} & getFrame(), deepLearningDetection(),videoSummarization()
\\ \hline
\textbf{Constraints} & cannot work without Frame class.
\\ \hline
\end{tabular}
\end{table}


 \FloatBarrier
\subsection{Patterns use viewpoint}
\FloatBarrier
\begin{figure}[h]
  \centering
  \includegraphics[width=20cm]{sequenceDiagram.png}
  \label{fig:2}
   \caption{Sequence Diagram}
\end{figure}
\subsubsection{Design Rationale}
You need to provide the design rationale for using these design patterns.

\subsection{Algorithm viewpoint}
The system consists of edge-fog layers, where each layer is responsible for a sequence of tasks that it completes and passes on to the next layer. 
\begin{itemize}
\item Sensors placed on street lights are introduced in this system as our edge nodes. The edge nodes are responsible for checking for camera failures. They make sure that the camera capturing is not interrupted quality-wise by any external sources. Moreover, edge nodes apply a face comparison algorithm to compare faces detected with the faces detected in previous captures. Also, a crime detection algorithm using YOLO is applied to acknowledge if a crime occurred. When a crime is detected, the nodes determine the current location of the incident and send it, with the video streams, to the next layer. 

\item Vehicles are proposed as the system’s nodes for the second layer, the first tier fog layer. The fog nodes in this layer are responsible for applying a video summarization approach which is the keyframe extraction technique. The goal of this approach is to decrease the computational power by eliminating repeated frames and choosing frames that have relevant information as keyframes. Moreover, a face blurring algorithm is applied to protect the privacy of the citizens. 

\item Summarized video frames are then passed to the second-tier fog layer, where the nodes are the traffic lights. Deep learning techniques are applied to determine if the contact between people was unproblematic or if an incident occurred. If a crime occurred, the algorithm determines the crime’s type. Moreover, deblurring and face detection get applied to expose the criminal's face.

\item The cloud layer, which is the last layer, collects all the data passed to it by the layers and triggers an alarm to police units. Moreover, it backs up the data received by the layers and contains an archive to store video streams from the edge layer.



\end{itemize}
\subsection{Interaction viewpoint}
\begin{figure}[h]
  \centering
  \includegraphics[width=19cm]{Activity diagram.jpeg}
  \label{fig:2}
   \caption{The workflow diagram of the framework}
\end{figure}


\newpage

\subsection{Interface viewpoint}
\subsubsection{Main Page}
The admin can create accounts for multiple users. Each user logs into the system with his email and password. Moreover, each user can log out of the system.
\begin{itemize}
    \item Login allows the user to enter the system using his email and password
    \item Registeration allows the admin to create account for the users.
    \item Logout allows the user to exit the system.
\end{itemize}

\subsubsection{Location Page}
The location page consists of a map which views the incidents location. Users can track incidents to take appropriate action.

\subsubsection{Alert Page}
The alert page consists of alerts triggered by the cloud layer. It has all the information collected needed by police units to track and capture criminals.

\section{Data Design}
\subsection{Data Description}
Our system handles and processes data in various ways as data needed are retrieved from either the application implemented for storing users’ data or from the edge layer of our system’s architecture. Moreover, Data received from our dataset to train our models is originally in CSV file format. Data get captured in the application from the forms available for the users to fill in. On the other hand, sensors (cameras) placed on the edge layer capture data in video format to get processed further. Information extracted from users will be stored in Cloud Firestore, a NoSQL database that will provide our application with flexibility and scalability.
\begin{figure}[h]
  \centering
  \includegraphics[width=15cm]{ERdiagram.jpg}
  \label{fig:2}
   \caption{ER Diagram}
\end{figure}

\newpage
\subsection{Dataset Description}
We have acquired a data set of surveillance videos published by the \textbf{ 'UCF Center for research in computer vision'} called  \textbf{ UCF-Crime}; It is a large-scale dataset of 128 hours of videos; it has 1900 video of many classes as Robbery, Stealing, Shooting ... etc. And it is widely used in any research or project that targets public safety and surveillance. Therefore we took some of those videos from the classes that meet our requirements. \ref{tab:ds} illustrates the dataset.

\FloatBarrier
\begin{table}[htbp]
\centering
\label{tab:ds}
\caption{Dataset Description}
\begin{tabular}{ | l | m{11cm}|} 
\hline
\textbf{Dataset Name} & UCF-Crime Dataset   \\ 
\hline
\textbf{Link} &  https://www.crcv.ucf.edu/projects/real-world/ \\ 
\hline
\textbf{Size} & 95.9 GB  \\ 
\hline
\textbf{Number of classes} & The dataset has 13 classes for anomaly events and normal class; we used 3 of them  \\
\hline
\textbf{Average Video frames} & 1000-1500 frames per video \\
\hline
\textbf{Number of videos used} & we used for fighting 10 videos,Robbery 10 videos,normal events 10 videos \\
\hline
\textbf{comments} & This data set has many videos in many places; we chose the suitable videos in streets. Also it didn't include kidnapping events; so we gathered 6 surveillance videos from Youtube with same video length and specifications as UCF-crime videos \\
\hline
\end{tabular}
\end{table}
 \FloatBarrier
\begin{figure}[h]
  \centering
  \includegraphics[width=5cm]{ForgetPass.PNG}
  \label{fig:2}
   \caption{Forget Password}
\end{figure}
\FloatBarrier
\begin{figure}[h]
  \centering
  \includegraphics[width=5cm]{profile.PNG}
  \label{fig:2}
   \caption{Profile}
\end{figure}
\FloatBarrier
\begin{figure}[h]
  \centering
  \includegraphics[width=5cm]{Alerts.PNG}
  \label{fig:2}
   \caption{Alerts Page}
\end{figure}
\FloatBarrier
\begin{figure}[h]
  \centering
  \includegraphics[width=5cm]{Logout.PNG}
  \label{fig:2}
   \caption{Logout Page}
\end{figure}

\subsection{Database design description}

\begin{figure}[h]
  \centering
  \includegraphics[width=15cm]{database surv.png}
  \label{fig:2}
   \caption{data base}
\end{figure}
The admin is responsible for creating the officers accounts and later the officers has the ability to login and to see their alerts panel in case any incident is detected.
The camera of each location is linked with the nearest police station.
\FloatBarrier
\FloatBarrier

\section{Human Interface Design}
\label{hid}
\subsection {User Interface}
The proposed system's application focuses on two main ends. User end and admin end. The application allows the admin to login and creates accounts for the police officer or security guards. Also, the admin can update and delete the user's account. The user will log in to the application with an email and password set by the admin. Also, the user can update his information and change his password. Moreover, the user will view the alerts sent by the system with images, location, and time.

Admin: Admin will log in through the application login page. Admin can add accounts by filling the fields in add account page with proper information.

User: The user will log in through the application login page. He will be able to edit the profile by updating the user's info through the profile page. Also, the user will view the alerts sent by the system with their details on the alert page.

\newpage
\subsection {Screen Images}
\FloatBarrier
\begin{figure}[h]
  \centering
  \begin{minipage}[b]{0.32\textwidth}
    \includegraphics[width=\textwidth]{HomePage.PNG}
    \caption{Home Page}
  \end{minipage}
  \hfill
  \begin{minipage}[b]{0.32\textwidth}
    \includegraphics[width=\textwidth]{Login.PNG}
    \caption{Login Page}
     
      \end{minipage}
  \begin{minipage}[b]{0.32\textwidth}
    \includegraphics[width=\textwidth]{AddAccount.PNG}
    \caption{Add new account}
  \end{minipage}
\end{figure}
\clearpage

\begin{figure}[h]

  \begin{minipage}[b]{0.5\textwidth}
    \includegraphics[width=5.0cm, height=10.0cm]{ForgetPass.PNG}
   \caption{Forget Password}
   \label{fig:2}
  \end{minipage}
    \begin{minipage}[b]{0.5\textwidth}
    \includegraphics[width=5.0cm, height=10.0cm]{profile.PNG}
    \label{fig:2}
    \caption{Profile}
  \end{minipage}
      \begin{minipage}[b]{0.5\textwidth}
    \label{fig:2}
    \includegraphics[width=5.0cm, height=10.0cm]{Alerts.PNG}
    \caption{Alerts Page}
  \end{minipage}
  \begin{minipage}[b]{0.5\textwidth}
    \includegraphics[width=5.0cm, height=10.0cm]{Logout.PNG}
     \label{fig:2}
   \caption{Logout Page}
  \end{minipage}
  
\label{label1}
\end{figure}

\FloatBarrier

\subsection {Screen Objects and Actions}
Our proposed system has two main stakeholders, the admin will login by his email and password then he is responsible to create accounts for police officers or security guards also has the power to edit, delete user’s accounts. Moreover, the second one is user, he will login with his email and password in the login page that was created by the admin, as he has the ability to edit or delete his information and can view alerts, location, and images sent by the system. Also, the user can make a new password if he forgets his password on forget password page.


\section{Requirements Matrix}
\begin{table}[htbp]
\centering
\caption{Requirements Matrix}
\label{tab:RM}
\begin{tabular}{|l|l|l|l|l|}
\hline
Req. ID  & Req Desc & Class & Test Cases ID & Status      \\ \hline
FR01     & camera failure (edge layer)     & camera & TC03, TC04, TC05    & Developed \\ \hline
FR03     & crime detection using YOLO (edge layer)     & camera & TC01, TC02    & Developed \\ \hline
FR04     & video summarization (fog tier 1)     & image & TC06   & In Progress   \\ \hline
FR05     & face blurring (fog tier 1)     & object & TC07   & Developed   \\ \hline
FR06     & deep learning detection (fog tier 2)     & image & TC08, TC09   & In Progress   \\ \hline
FR07     & face detection (fog tier 2)     & object & TC10   & Developed   \\ \hline
FR09     & login     & officers & TC11,TC12   & Developed   \\ \hline
FR10     & create account     & Admin & TC13,TC14,TC15   & Developed   \\ \hline
\end{tabular}
\end{table}


\section{APPENDICES}
Fog computing: A mid-layer between edge devices and the cloud, at which data processing occurs. 
\\
\\
Edge computing: An architecture at which data gets processed on devices rather than the cloud to provide low latency.
\newpage
\subsection{Github}
\begin{figure}[h]
  \centering
  \includegraphics[width=18cm]{Screenshot (403).png}
  \label{fig:2}
   \caption{GitHub repository}
\end{figure}

\subsection{ Other appendices as appropriate}
Optional section.

\printbibliography

\end{document}
