\documentclass[12pt]{article}
\usepackage{graphicx}
\usepackage[utf8]{inputenc}
\usepackage{makecell}
\usepackage[margin=1in]{geometry}
\usepackage[T1]{fontenc}
\usepackage{mathptmx}
\usepackage[normalem]{ulem}

\renewcommand\theadalign{bc}
\renewcommand\theadfont{\bfseries}
\renewcommand\theadgape{\Gape[4pt]}
\renewcommand\cellgape{\Gape[4pt]}

%opening
\title{Test Plan for Intelligent Surveillance system for smart cities }
\author{
Nour Ahmed, Nour El Hoda Hisham, Mariam Hesham, Samiha Hesham, Sandra Fares\\
Supervised by: Eng Lobna Mostafa, Dr.Islam Tharwat
}

\begin{document}
\maketitle

\begin{table}[htp]
\caption{Document version history}
\begin{center}
\begin{tabular}{|c|c|l|}
\hline
\thead{Version}    & \thead{Date} & \thead{Reason for Change}  \\ \hline
1.0 & 27-Mar-2021   & \makecell[l]{Test Plan First version is defined.}   \\ \hline
1.1 & 29-Mar-2021   & \makecell[l]{Non-functional Requirements testing is added}   \\ \hline

\end{tabular}
\end{center}
\end{table}

\begin{table}[htp]
\begin{tabular}{cc}
\thead{GitHub:}    & {https://github.com/SandraFW/intelligent-Surveillance-for-smart-cities}   
\end{tabular}
\end{table}

\pagebreak
\tableofcontents
\pagebreak


\section{Introduction}

\subsection{Purpose}
 This document aims to describe the approaches and methodologies that will be used to test the system functions performance and whether it meets our requirements mentioned in the Software requirements specification paper or not.

\subsection{Scope}
The scope of this test plan is to make sure that all the requirements proposed in the software specification requirements are met and developed correctly.  Any Test plan changes will be done, they will be documented in this document by changing the version numbers of the document. 

\section{Test Scenario: check crime detection's functionality}\label{sec:TSx}

The crime detection function FR03 , It depends on the camera captured the video successfully then it must detect either the video streams has any abnormal behaviour or normal in order to send the video stream to the following function.  
\subsection{Test Cases}
\begin{table}[h]
\caption{Test Cases for Scenario 1}
\label{tab:TC1}
\begin{tabular}{|p{0.1\linewidth}|p{0.3\linewidth}|p{0.1\linewidth}|p{0.2\linewidth}|p{0.2\linewidth}|}
\hline
Test Case ID & Test Case Desc & Functional Req Code & Test Data & Expected Result \\ \hline
TC01  & an incident got detected & FR03 &  extracted frames received from videos captured by the cameras  & the incident is annotated with a boundary box and the video is sent to be summarized in the next layer           \\ \hline
TC02 & no incident is detected   & FR03    &  extracted frames received from videos captured by the cameras     & video is sent to the archive in the cloud layer           \\ \hline

\end{tabular}
\end{table}
\newpage
\section{Test Scenario: Check for camera failure}\label{sec:TSy}
As our system depends heavily on data captured by the sensors (cameras), it is essential to check the availability of sensors and make sure that their functionality is not interrupted by any external source.

\subsection{Test Cases}
\begin{table}[h]
\caption{Test Cases for Scenario 1}
\label{tab:TC1}
\begin{tabular}{|p{0.1\linewidth}|p{0.3\linewidth}|p{0.1\linewidth}|p{0.2\linewidth}|p{0.2\linewidth}|}
\hline
Test Case ID & Test Case Desc & Functional Req Code & Test Data & Expected Result \\ \hline
TC03  & a black screen appeared while capturing data & FR01 &  video captured by cameras   &  an alert is triggered                \\ \hline
TC04 & a green screen appeared while capturing data   & FR01    &  video captured by cameras     &  an alert is triggered           \\ \hline
TC05 & an unfocused camera view  & FR01    &  video captured by cameras     &  quality enhanced          \\ \hline

\end{tabular}
\end{table}


\section{Test Scenario: Check for video summarization functionality}\label{sec:TSy}
It is beneficial to extract the needed information instead of processing irrelevant data, especially when our goal is to decrease computational power. That is why video summarization plays a huge goal in this system. It's important to make sure that it successfully does its role.
\subsection{Test Cases}
\begin{table}[h]
\caption{Test Cases for Scenario 1}
\label{tab:TC1}
\begin{tabular}{|p{0.1\linewidth}|p{0.3\linewidth}|p{0.1\linewidth}|p{0.2\linewidth}|p{0.2\linewidth}|}
\hline
Test Case ID & Test Case Desc & Functional Req Code & Test Data & Expected Result \\ \hline
TC06  & a video is received from the edge layer & FR04 &  output video from the crime detection algorithm that specifies an incident    &  keyframes extracted and video is summarized                \\ \hline

\end{tabular}
\end{table}

\newpage
\section{Test Scenario: Check for face blurring functionality}\label{sec:TSy}
After applying video summarization, video streams are passed to a face blurring algorithm to protect the privacy of citizens.
\subsection{Test Cases}
\begin{table}[h]
\caption{Test Cases for Scenario 1}
\label{tab:TC1}
\begin{tabular}{|p{0.1\linewidth}|p{0.3\linewidth}|p{0.1\linewidth}|p{0.2\linewidth}|p{0.2\linewidth}|}
\hline
Test Case ID & Test Case Desc & Functional Req Code & Test Data & Expected Result \\ \hline
TC07  & a video is received from the summarization function & FR05 & summarized video received from the video summarization function    &  blurred faces                \\ \hline

\end{tabular}
\end{table}

\section{Test Scenario: Check for Deep learning detection functionality}\label{sec:TSy}
After applying video summarization and face blurring, video streams are passed to the detection model, placed in the fog layer, so that it could detect and classify the crime use case that took place. 
\subsection{Test Cases}
\begin{table}[h]
\caption{Test Cases for Scenario 1}
\label{tab:TC1}
\begin{tabular}{|p{0.1\linewidth}|p{0.3\linewidth}|p{0.1\linewidth}|p{0.2\linewidth}|p{0.2\linewidth}|}
\hline
Test Case ID & Test Case Desc & Functional Req Code & Test Data & Expected Result \\ \hline
TC08  & an anomaly got detected & FR06 &   frames extracted from video received from the face blurring function in the first tier fog layer   & video is extracted to apply deblurring                \\ \hline
TC09 & anomaly is not detected   & FR06    &  frames extracted from video received from the face blurring function in the first tier fog layer      &  video is sent to the archive in the cloud layer         \\ \hline

\end{tabular}
\end{table}
\newpage
\section{Test Scenario: Check for face detection's functionality}\label{sec:TSy}
If an anomaly got detected, the faces get deblurred and a face detection algorithm gets applied.
\subsection{Test Cases}
\begin{table}[h]
\caption{Test Cases for Scenario}
\label{tab:TC1}
\begin{tabular}{|p{0.1\linewidth}|p{0.3\linewidth}|p{0.1\linewidth}|p{0.2\linewidth}|p{0.2\linewidth}|}
\hline
Test Case ID & Test Case Desc & Functional Req Code & Test Data & Expected Result \\ \hline
TC10  & face detected & FR07 &  video frames that are extracted from the deep learning detection method.   & trigger an alarm                \\ \hline

\end{tabular}
\end{table}

\section{Test Scenario: application's login functionality}\label{sec:TSy}
\subsection{Test Cases}
\begin{table}[h]
\caption{Test Cases for Scenario}
\label{tab:TC1}
\begin{tabular}{|p{0.1\linewidth}|p{0.3\linewidth}|p{0.1\linewidth}|p{0.2\linewidth}|p{0.2\linewidth}|}
\hline
Test Case ID & Test Case Desc & Functional Req Code & Test Data & Expected Result \\ \hline
TC11  & user entered an incorrect email or password & FR09 & string   & user is not allowed to proceed with the application               \\ \hline
TC12 & user entered a correct email and password   & FR09    &  string      &  user can proceed with the application and view alerts          \\ \hline

\end{tabular}
\end{table}
\newpage
\section{Test Scenario: creating a new account}\label{sec:TSy}
\subsection{Test Cases}
\begin{table}[h]
\caption{Test Cases for Scenario}
\label{tab:TC1}
\begin{tabular}{|p{0.1\linewidth}|p{0.3\linewidth}|p{0.1\linewidth}|p{0.2\linewidth}|p{0.2\linewidth}|}
\hline
Test Case ID & Test Case Desc & Functional Req Code & Test Data & Expected Result \\ \hline
TC13  & Admin creates, for a user, an account that already exists & FR10 & string   & new account is not added and an alert message appears               \\ \hline
TC14 & Admin enters wrong data format   & FR10    &  string      &  validation messages appear         \\ \hline
TC15 & Admin enters correctly formatted data   & FR10    &  string      &  a new account got created         \\ \hline

\end{tabular}
\end{table}
\section{Test Scenario: Non-functional Requirements testing}
After deploying all the functional requirements, the non functional requirements mentioned in software specification document will be tested.And then it will be compared with other systems introduced to check if our work accomplished its target.
\subsection{Scalability testing}
the system will undergo some scalability tests using ifogsims simulator to check Scalability attributes as:\begin{itemize}
    \item Response Time
    \item CPU Usage
    \item Network Usage
    \item Performance under load
\end{itemize}
and then it will be compared with other similar systems performance.

\subsection{Security testing}
We will apply a security test as vulnerability scanning, security scanning,etc to identify the threats in the system and measure its potential vulnerabilities, so the threats can be encountered.

\subsection{Availability Testing}
One of the most non functional testing as the system will work continuously for long period of times.We must ensure that the system components do not crash.
\end{document}
