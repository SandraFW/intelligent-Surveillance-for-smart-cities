\documentclass[12pt]{article}
\usepackage{graphicx}
\usepackage[utf8]{inputenc}
\usepackage{makecell}
\usepackage[margin=1in]{geometry}
\usepackage[T1]{fontenc}
\usepackage{mathptmx}
\usepackage[normalem]{ulem}

\renewcommand\theadalign{bc}
\renewcommand\theadfont{\bfseries}
\renewcommand\theadgape{\Gape[4pt]}
\renewcommand\cellgape{\Gape[4pt]}

%opening
\title{Test Plan for Intelligent Surveillance system for smart cities }
\author{
Nour Ahmed, Nour El Hoda Hisham, Mariam Hesham, Samiha Hesham ,Sandra Fares\\
Supervised by: Eng Lobna Mostafa, Dr.Islam Tharwat
}

\begin{document}
\maketitle

\begin{table}[htp]
\caption{Document version history}
\begin{center}
\begin{tabular}{|c|c|l|}
\hline
\thead{Version}    & \thead{Date} & \thead{Reason for Change}  \\ \hline
1.0 & 25-Jan-2021   & \makecell[l]{Test Plan First version is defined.}   \\ \hline
1.1 & 2-Feb-2021   & \makecell[l]{Test Scenario is Added.} \\ \hline
1.3 & 5-Feb-2021   & \makecell[l]{Test case is added.} \\
\hline
\end{tabular}
\end{center}
\end{table}

\begin{table}[htp]
\begin{tabular}{cc}
\thead{GitHub:}    & {https://github.com/SandraFW/intelligent-Surveillance-for-smart-cities}   
\end{tabular}
\end{table}

\pagebreak
\tableofcontents
\pagebreak


\section{Introduction}

\subsection{Purpose}
The Test Plan has been created to facilitate communication among the team members. This document describes approaches and methodologies that will be applied to the system functions and the whole system. This document will identify what the test deliverables will be.

\subsection{Scope}
The scope of this test plan is to make sure that all the requirements proposed in the software specification requirements are met and developed correctly.  Any Test plan changes will be done, they will be documented in this document by changing the version numbers of the document. 

\section{Test Scenario: check crime detection's functionality}\label{sec:TSx}

The crime detection function FR03 , It depends on the camera captured the video successfully then it must detect either the video streams has any abnormal behaviour or normal in order to send the video stream to the following function.  
\subsection{Test Cases}
\begin{table}[h]
\caption{Test Cases for Scenario 1}
\label{tab:TC1}
\begin{tabular}{|p{0.1\linewidth}|p{0.3\linewidth}|p{0.1\linewidth}|p{0.2\linewidth}|p{0.2\linewidth}|}
\hline
Test Case ID & Test Case Desc & Functional Req Code & Test Data & Expected Result \\ \hline
TC01  & an incident got detected & FR03 &  video frames  & the incident is annotated with a boundary box and the video is sent to be summarized in the next layer           \\ \hline
TC02 & no incident is detected   & FR03    &  video frames     & video is sent to the archive in the cloud layer           \\ \hline

\end{tabular}
\end{table}
\newpage
\section{Test Scenario: Check for camera failure}\label{sec:TSy}
As our system depends heavily on data captured by the sensors (cameras), it is essential to check the availability of sensors and make sure that their functionality is not interrupted by any external source.

\subsection{Test Cases}
\begin{table}[h]
\caption{Test Cases for Scenario 1}
\label{tab:TC1}
\begin{tabular}{|p{0.1\linewidth}|p{0.3\linewidth}|p{0.1\linewidth}|p{0.2\linewidth}|p{0.2\linewidth}|}
\hline
Test Case ID & Test Case Desc & Functional Req Code & Test Data & Expected Result \\ \hline
TC03  & a black screen appeared while capturing data & FR01 &  video   &  an alert is triggered                \\ \hline
TC04 & a green screen appeared while capturing data   & FR01    &  video      &  an alert is triggered           \\ \hline
TC05 & an unfocused camera view  & FR01    &  video      &  quality enhanced          \\ \hline

\end{tabular}
\end{table}


\section{Test Scenario: Check for video summarization functionality}\label{sec:TSy}
It is beneficial to extract the needed information instead of processing irrelevant data, especially when our goal is to decrease computational power. That is why video summarization plays a huge goal in this system. It's important to make sure that it successfully does its role.
\subsection{Test Cases}
\begin{table}[h]
\caption{Test Cases for Scenario 1}
\label{tab:TC1}
\begin{tabular}{|p{0.1\linewidth}|p{0.3\linewidth}|p{0.1\linewidth}|p{0.2\linewidth}|p{0.2\linewidth}|}
\hline
Test Case ID & Test Case Desc & Functional Req Code & Test Data & Expected Result \\ \hline
TC06  & a video is received from the edge layer & FR04 &  video frames   &  keyframes extracted                \\ \hline

\end{tabular}
\end{table}

\newpage
\section{Test Scenario: Check for face blurring functionality}\label{sec:TSy}
After applying video summarization, video streams are passed to a face blurring algorithm to protect the privacy of citizens.
\subsection{Test Cases}
\begin{table}[h]
\caption{Test Cases for Scenario 1}
\label{tab:TC1}
\begin{tabular}{|p{0.1\linewidth}|p{0.3\linewidth}|p{0.1\linewidth}|p{0.2\linewidth}|p{0.2\linewidth}|}
\hline
Test Case ID & Test Case Desc & Functional Req Code & Test Data & Expected Result \\ \hline
TC07  & a video is received from the summarization function & FR05 &  video   &  blurred faces                \\ \hline

\end{tabular}
\end{table}

\section{Test Scenario: Check for Deep learning detection functionality}\label{sec:TSy}
After applying video summarization and face blurring, video streams are passed to the detection model, placed in the fog layer, so that it could detect and classify the crime use case that took place. 
\subsection{Test Cases}
\begin{table}[h]
\caption{Test Cases for Scenario 1}
\label{tab:TC1}
\begin{tabular}{|p{0.1\linewidth}|p{0.3\linewidth}|p{0.1\linewidth}|p{0.2\linewidth}|p{0.2\linewidth}|}
\hline
Test Case ID & Test Case Desc & Functional Req Code & Test Data & Expected Result \\ \hline
TC08  & an anomaly got detected & FR06 &  video frames   & video is extracted to apply deblurring                \\ \hline
TC09 & anomaly is not detected   & FR06    &  video frames      &  video is sent to the archive in the cloud layer         \\ \hline

\end{tabular}
\end{table}
\newpage
\section{Test Scenario: application's login functionality}\label{sec:TSy}
\subsection{Test Cases}
\begin{table}[h]
\caption{Test Cases for Scenario 1}
\label{tab:TC1}
\begin{tabular}{|p{0.1\linewidth}|p{0.3\linewidth}|p{0.1\linewidth}|p{0.2\linewidth}|p{0.2\linewidth}|}
\hline
Test Case ID & Test Case Desc & Functional Req Code & Test Data & Expected Result \\ \hline
TC10  & user entered an incorrect email or password & FR09 & string   & user is not allowed to proceed with the application               \\ \hline
TC11 & user entered a correct email and password   & FR09    &  string      &  user can proceed with the application and view alerts          \\ \hline

\end{tabular}
\end{table}

\section{Test Scenario: creating a new account}\label{sec:TSy}
\subsection{Test Cases}
\begin{table}[h]
\caption{Test Cases for Scenario 1}
\label{tab:TC1}
\begin{tabular}{|p{0.1\linewidth}|p{0.3\linewidth}|p{0.1\linewidth}|p{0.2\linewidth}|p{0.2\linewidth}|}
\hline
Test Case ID & Test Case Desc & Functional Req Code & Test Data & Expected Result \\ \hline
TC12  & Admin creates, for a user, an account that already exists & FR10 & string   & new account is not added and an alert message appears               \\ \hline
TC13 & Admin enters wrong data format   & FR10    &  string      &  validation messages appear         \\ \hline
TC14 & Admin enters correctly formatted data   & FR10    &  string      &  a new account got created         \\ \hline

\end{tabular}
\end{table}

\end{document}
